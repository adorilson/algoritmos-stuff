\documentclass[12pt]{article}
\usepackage{amsmath,amsfonts,amssymb,amsthm}
\usepackage{color,epsfig}
\usepackage[brazil]{babel}
\usepackage[utf8]{inputenc}
\usepackage{graphicx}
\usepackage[top=3cm,left=3cm,right=2cm,bottom=3cm]{geometry}
\usepackage{verbatim}
\usepackage{algorithm}
\usepackage{algorithmic}
\usepackage{enumerate}
\usepackage{tikz}
\usetikzlibrary{arrows}

\usepackage{listings}
\lstset{
  language=C++,
  basicstyle=\ttfamily\small,
  keywordstyle=\color{blue},
  stringstyle=\color{green},
  commentstyle=\color{red},
  extendedchars=true,
  showspaces=false,
  showstringspaces=false,
  numbers=left,
  numberstyle=\tiny,
  breaklines=true,
  breakautoindent=true,
  captionpos=b,
  xleftmargin=0pt,
}


% Title Page
\title{Exercícios Complementares sobre Estratégia Gulosa}
\author{Adorilson Bezerra de Araújo}
\date{Janeiro de 2015}

\begin{document}

\begin{center}
{\large\sc Universidade Federal do Rio Grande do Norte} \\
{\large\sc Centro de Ciências Exatas e da Terra} \\
{\large\sc Departamento de Informática e Matemática Aplicada}

\vspace*{0.5cm}

{\sc DIM0806.1 -- Estruturas de Dados e Algoritmos -- Turma 1}\\
{\sc Prof. David Déharbe e Prof. Selan Rodrigues}

\vspace*{0.2cm}

{\sc Adorilson Bezerra de Araújo}

\vspace*{0.2cm}

{\bf Exercícios Complementares sobre Estratégia Gulosa}
\end{center}

%!TEX root = report.tex
\section{Greedy}\label{sec:greedy}

\subsection{Exercício 23}\label{sec:exer23}

\begin{tikzpicture}[thick, framed]
  \draw[|-|] (0, 1.5) -- (6, 1.5)
    node[pos=0.5,above]{$\textsc{A}$};
  \draw[|-|] (1, 1) -- (4, 1)
        node[pos=0.5,above]{$\textsc{B}$};
  \draw[|-|] (3, 0.5) -- (5, 0.5)
        node[pos=0.5,above]{$\textsc{C}$};
    \draw[|-|] (3, 0) -- (8, 0)
        node[midway,above]{$\textsc{D}$};
    \draw[|-|] (4, -0.5) -- (7, -0.5)
        node[pos=0.5,above]{$\textsc{E}$};
    \draw[|-|] (5, -1.0) -- (9, -1.0)
        node[pos=0.5,above]{$\textsc{F}$};
    \draw[|-|] (6, -1.5) -- (10, -1.5)
        node[pos=0.5,above]{$\textsc{G}$};
    \draw[|-|] (8, -2.0) -- (11, -2.0)
        node[pos=0.5,above]{$\textsc{H}$};
    \draw (5.5,2) node[above,xshift=0.7cm]{$ \textsc{Intervalos}$};%
    % Axis
    \draw (0, -2.5) -- (12, -2.5)
      node[pos=0.5,below]{$\textsc{Tempo}$};
    % Note that the snaked line is drawn to 11.1 to force
    % TikZ to draw the final tick.
    \draw[snake=ticks,segment length=1cm] (0, -2.5) -- (12, -2.5);
\end{tikzpicture}

Considerando os intervalos apresentados acima, caso utilizemos uma estratégia
earliest starts first (a) seriam selecionados os intervalos A e G. Na estratégia
shortest duration first (b) os intervalos C e H seriam escolhidos. Já para a 
estratégia earliest finishes first (c) os intervalos seriam B, E e H. Portanto,
a estratégia earliest finishes first é a estratégia ótima para esse tipo de problema.
Intuitiva, podemos perceber que isso ocorre por que quanto antes o slot de tempo
for liberado para o próximo intervalo, melhor.

As estratégias a) e b) poderiam 
ainda ser piores, caso o intervalo A ocupasse todo o tempo, assim seria incompatível
como todos os demais, mas por ser o primeiro a começar, seria selecionado. Ou ainda
se tivéssemos um intervalo pequeno mas que fizesse intercecção como outros dois, que
não possuíssem intercecção entre si. Dessa forma, esse intervalo pequeno seria
selecionado, e somente ele, em detrimentos dos outros dois.

O algoritmo usando a estratégia earliest finishes first pode ser vista abaixo.

\begin{algorithm}
  \caption{Determina o maior conjunto possível de intervalos sem que haja sobreposição
  de tempos entres eles}
  \begin{algorithmic}
    \REQUIRE $n, S=[s_{0}, s_{1}, \cdots, s_{n}]$, $F=[f_{0}, f_{1}, \cdots, f_{n}]$
    \STATE Ordenar os intervalos por ordem crescente do tempo de finalização
    \STATE $A \leftarrow \emptyset$
    \FOR{$j = 0$ to n}
      \IF {intervalo j é compatível com A}
            \STATE $A \leftarrow A \cup \lbrace j \rbrace$
      \ENDIF
    \ENDFOR
    \RETURN A
  \end{algorithmic}
\end{algorithm}

\newpage
Esse algoritmo pode ser implementado em $\theta(n log n)$, desde que se respeitem alguns
critérios:
\begin{enumerate}
  \item Manter uma referencia para j*, que foi o ultimo intervalor adicionado a A
  \item O intervalo j é compatível com A se $s_{j} \ge f_{j*}$
  \item A ordenação dos intervalores ser feita em um $\theta(n log n)$
\end{enumerate}

\subsection{Exercício 24}\label{sec:exer24}
Sim. Basta que se defina o critério de qual vertice será maior ou menor que outro.
Se irá ser considerado o valor real dele (valor negativo) ou o valor em módulo.

\subsection{Exercício 25}\label{sec:exer25}
Não. O algoritmo de Prim só trabalha com grafos conectados com pesos nos vertices.
Para grafos sem pesos nos vertices, outro(s) algoritmo(s) deve(m) ser aplicado(s).

\subsection{Exercício 26}\label{sec:exer26}

Consideremos o grafo abaixo:

\begin{tikzpicture}[->, >=stealth', shorten >=1pt, auto, node distance=3cm, thick,
    main node/.style={circle,fill=white, draw,minimum size=0.15cm,inner sep=0pt]},
	]
	
	\node[main node] (1) [] {A};
	\node[main node] (2) [right of=1] {B};
	\node[main node] (3) [above left of=2] {C};
	\node[main node] (4) [below left of=2] {D};
	
	\path[every node/.style={font=\sffamily\small}]
	(1) edge node {1} (2)
	(2) edge [bend right] node[right] {1} (3)
	(1) edge [bend left] node[left] {0} (3)
	(1) edge [bend right] node[left] {99} (4)
	(4) edge [bend right] node[right] {-300} (2)	
	;
\end{tikzpicture}

Assumindo que os vertices são direcionados conforme as setas e deseja-se obter o menor
caminho para nó C, o algoritmo de Dijkstra irá trabalhar da seguinte forma:

\begin{enumerate}
  \item Primeiro, define d(A) para zero e as demais distâncias para o infinito
  \item Partindo do A em direção aos demais nós, definimos d(B)=1, d(C)=0 e d(D)=99 
  \item Não há vertices ligando o nó C a nenhum outro
  \item Partindo de B, também não há mudança (já que 2 é maior que zero)
  \item Finalmente, partindo de D, d(B) será alterado para -201
\end{enumerate}

Ao final da execução do algoritmo, contudo, d(C) ainda é 0, mesmo que o menor caminho
para C seja -200. Isso ocorre porque se eu tenho um número (d(D)) maior que outro (d(C))
e adiciono a ele um outro número positivo (o esperado pelo algoritmo), necessariamente
esse valor ainda será maior. Portanto, o menor caminho para C, necessariamente não será
passando por D, já que um outro menor, até aquele momento já foi identificado.

\subsection{Exercício 27}\label{sec:exer27}

\begin{enumerate}[a)]
\item \begin{table}[ht]
    \begin{center}
      \begin{tabular}{lccccc}
símbolo & A & B & C & D & - \\
        \hline
frequência & 0.4 & 0.1 & 0.2  & 0.15     & 0.15 \\
\hline
código & 1 & 010   & 011   & 000     & 001   \\
      \end{tabular}
    \end{center}
  \end{table}

    \item 1010101110101000
    \item ADACA\_B
\end{enumerate}


\subsection{Exercício 28}\label{sec:exer28}


\subsection{Exercício 29}\label{sec:exer29}

\subsection{Exercício 30}\label{sec:exer30}

\subsection{Exercício 31}\label{sec:exer31}


\end{document}
