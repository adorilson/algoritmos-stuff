%!TEX root = report.tex

\newpage

\section{Divide and Conquer}\label{sec:divide_and_conquer}

\subsection{Exercício 9}\label{sec:exer9}

\begin{algorithm}
  \caption{Retorna o indice do maior elementos de uma lista usando uma estratégia divisão e conquista}
  \begin{algorithmic}
    \REQUIRE $A=[a_{0}, a_{1}, \cdots, a_{n}]$, n
    \ENSURE Retorna i onde A[i] é o maior elemento em A
    \STATE Function FindMax(A, n)
      \RETURN FindMaxAux(A, 0, n-1, 0)
    \STATE EndFunction
    
    \STATE Function FindMaxAux(A, left, right, indexMax)
      \IF {$left < right$}
        \STATE $m \leftarrow $(left + right)/2
        \STATE $x \leftarrow $FindMaxAux(A, left, m, indexMax)
        \STATE $y \leftarrow $FindMaxAux(A, m+1, right, x)
        \RETURN y
      \ELSE
        \IF {$A[left]>A[indexMax]$}
          \RETURN left
        \ELSE
          \RETURN indexMax
        \ENDIF
      \ENDIF
    \STATE EndFunction
  \end{algorithmic}
\end{algorithm}

\begin{enumerate}[a)]
  \item Será o menor índice
  \item Brute-force é $\Theta(n)$, Divide-and-conquer é $\Theta(n\log{}n)$
\end{enumerate}

\newpage
\subsection{Exercício 10}\label{sec:exer10}

\begin{algorithm}
  \caption{Retorna o maior e o menor elementos de um conjunto}
  \begin{algorithmic}
    \REQUIRE $A=[a_{0}, a_{1}, \cdots, a_{n}]$, n
    \ENSURE Retorna B onde B[0] e B[1] são, respectivamente, o menor e o maior elementos de A
    \STATE Function FindSmallestAndLargest(A, n)
      \STATE $B \leftarrow $int[IntMax, IntMin]
      \STATE FindAux(A, 0, n-1, B)
      \RETURN B
    \STATE EndFunction
    
    \STATE Function FindAux(A, left, right, *B)
      \IF {$left < right$}
        \STATE $m \leftarrow $(left + right)/2
        \STATE FindAux(A, left, m, B)
        \STATE FindAux(A, m+1, right, B)
      \ELSE
        \IF {$A[left]<B[0]$}
          \STATE $B[0] \leftarrow $A[left]
        \ENDIF
        \IF {$A[left]>B[1]$}
          \STATE $B[1] \leftarrow $A[left]
        \ENDIF
      \ENDIF
    \STATE EndFunction
  \end{algorithmic}
\end{algorithm}



\subsection{Exercício 11}\label{sec:exer11}

\begin{enumerate}[a)]
  \item T(n) = 4T(n/2)+n, T(1)=1 \\
  $f(n) \in \Theta(n) \therefore d=1$ \\
  $a = 4, b = 2 \therefore a>b^{d}$ \\
  $T(n) \in \Theta(n\log_{2}4) = \Theta(2n) \therefore \Theta(n)$
  \item T(n) = 4T(n/2)+$n^{2}$, T(1)=1 \\
  $f(n) \in \Theta(n^{2}) \therefore d=2$ \\
  $a = 4, b = 2 \therefore a=b^{d}$ \\
  $T(n) \in \Theta(n^{d} \log{}n) = \Theta(n^{2} \log{}n)$
  \item T(n) = 4T(n/2)+$n^{3}$, T(1)=1 \\
  $f(n) \in \Theta(n^{3}) \therefore d=3$ \\
  $a = 4, b = 2 \therefore a<b^{d}$ \\
  $T(n) \in \Theta(n^{d}) = \Theta(n^{3})$
\end{enumerate}

\subsection{Exercício 12}\label{sec:exer12}

\begin{lstlisting}
void merge( int* a , int* b , int e , int m , int d )
{
  int t, sl, sr;
  t = 0;
  sl = e;
  sr = m + 1;
  while(sl <= m && sr <= d){
    if(a[sl]<=a[sr]){
      b[t] = a[sl];
      sl = sl + 1;
    }
    else{
      b[t] = a[sr];
      sr = sr + 1;
    }
    t = t + 1;
  }
  while(sl <= m){
    b[t] = a[sl];
    sl = sl + 1;
    t = t + 1;
  }
  while(sr <= d){
    b[t] = a[sr];
    sr = sr + 1;
    t = t + 1;
  }
  for (int i = 0 ; i < (d - e + 1); i++){
    a[e + i] = b[i];
  }
  return ;
}

void mergesort( int* a , int size )
{  
  for(int i = 1; i < size ; i = i*2){
      for(int e = 0 ; e < (size-i); e += i*2){
          int m = e + i - 1;
          int aux1 = e + i*2 -1;
          int aux2 = size - 1;
          int d;
          if(aux1 <= aux2){
             d = aux1;
          }
          else{
            d = aux2;
          }
          int *b = new int[d-e+1];
          merge(a, b, e, m, d);
          delete b;
      } 
  }
  return ;
}
\end{lstlisting}

\subsection{Exercício 13}\label{sec:exer13}

Consideremos o conjunto A=[9a, 0a, 1a, 3, 0b, 1b, 9b] e o algoritmo QuickSort
abaixo:

\begin{algorithm}
  \caption{Algoritmo QuickSort}
  \begin{algorithmic}
    \STATE Function QuickSort(A, left, right)
      \STATE $S \leftarrow $partition(A, left, right)
      \STATE QuickSort(A, left, S-1)
      \STATE QuickSort(A, S+1, right)
    \STATE EndFunction
  \end{algorithmic}
\end{algorithm}

Teremos a seguinte pilha de chamadas: \\
\begin{tabbing}
QuickSort(A, 0, 7) \\
  \hspace{1em}$S \leftarrow $partition(A, 0, 7) // Retorna 7 e A=[9b, 0a, 1a, 3, 0b, 1b, 9a] \\
  \hspace{1em}QuickSort(A, 0, 6) // expandindo abaixo $\cdots $\\
  \hspace{1em}QuickSort(A, 8, 7)
\end{tabbing}

\begin{tabbing}
QuickSort(A, 0, 6) \\
  \hspace{1em}$S \leftarrow $partition(A, 0, 6) // Retorna 6 e A=[1b, 0a, 1a, 3, 0b, 9b, 9a] \\
  \hspace{1em}QuickSort(A, 0, 5) //expandindo abaixo $\cdots$ \\
  \hspace{1em}QuickSort(A, 7, 6)
\end{tabbing}

\begin{tabbing}
QuickSort(A, 0, 5) \\
  \hspace{1em}$S \leftarrow $partition(A, 0, 5) // Retorna 2 e A=[0b, 0a, 1b, 3, 1a, 9b, 9a] \\
  \hspace{1em}QuickSort(A, 0, 1)  \\
  \hspace{1em}QuickSort(A, 3, 5)
\end{tabbing}

\begin{tabbing}
\hspace{1em} \vdots
\end{tabbing}

Quando as chamadas QuickSort(A, 8, 7) e QuickSort(A, 7, 6) forem executadas nada sera feito, pois
left $>$ right (caso base da função partition). Assim, 9a e 9b já estão em seus locais 
definitivos e i' $>$ j', portanto o QuickSort é um algoritmo instável.

\subsection{Exercício 14}\label{sec:exer14}

\lstinputlisting[language=Python]{../q14.py}


\subsection{Exercício 15}\label{sec:exer15}

\begin{algorithm}
  \caption{Algoritmo LeafCounter}
  \begin{algorithmic}
    \STATE Function LeafCounter(T)
      \IF {$T=\varnothing$}
        \RETURN 0
      \ENDIF
      \IF {$T_{left}=\varnothing$ and $T_{right}=\varnothing$}
        \RETURN 1
      \ENDIF
      \RETURN LeafCounter($T_{left}$) + LeafCounter($T_{right}$)
    \STATE EndFunction
  \end{algorithmic}
\end{algorithm}

\newpage
\subsection{Exercício 16}\label{sec:exer16}

\begin{algorithm}
  \caption{OneDimensionClosestPair}
  \begin{algorithmic}
    \STATE Function OneDimensionClosestPair(Points, n)
      \IF {$n \le 3$}
        \RETURN $min(|D_1 - D_2|, |D_2 - D_3|)$
      \ELSE
        \STATE copiar os primeiros \big\lceil$n/2$\big\rceil pontos de Points para $P_l$
        \STATE copiar os restantes \big\lfloor$n/2$\big\rfloor pontos de Points para $P_r$
        \STATE $d_l \leftarrow OneDimensionClosestPair(P_l$, \big\lceil$n/2$\big\rceil)
        \STATE $d_r \leftarrow OneDimensionClosestPair(P_r$, \big\lfloor$n/2$\big\rfloor)
        \STATE $m_l \leftarrow P_l$[\big\lceil$n/2$\big\rceil-1] {//ultimo elemento de $P_l$}
        \STATE $m_r \leftarrow P_r[0]$ {//primeiro elemento de $P_r$}
        \STATE $m \leftarrow |m_l - m_r|$
        \STATE $d \leftarrow min(d_l, d_r, m)$
        \RETURN d
      \ENDIF
    \STATE EndFunction
  \end{algorithmic}
\end{algorithm}

T(n) = 2T(n/2)+n, T(1)=1 \\
  $f(n) \in \Theta(n) \therefore d=1$ \\
  $a = 2, b = 2 \therefore a=b^{d}$ \\
  $T(n) \in \Theta(n\log_{}n)$

