%!TEX root = report.tex

\section{Exhaustive Search}\label{sec:exhaustive_force}

\subsection{Exercício 5}\label{sec:exer5}
Consideremos o seguinte conjunto A=[1, 5, 2, 3, 4, 5]. Para resolver o problema da
partição, podemos seguir os passos abaixo:
\begin{enumerate}
  \item Calcular a soma total dos elementos. No caso, resulta em 20.
  \item Calcular a metade (no caso será 10). Esse será o valor da soma de cada
  subconjunto.
  \item Achar uma combinação de elementos cuja soma seja 10. Isso pode ser feito
  com o algoritmo abaixo.
\end{enumerate}

\begin{algorithm}
  \caption{Determina se existe uma combinação de elementos cuja soma seja a desejada}
  \begin{algorithmic}
    \REQUIRE $A=[a_{0}, a_{1}, \cdots, a_{n}]$,$usados=[a_{0}, a_{1}, \cdots, a_{n}]$,
    inicio, fim e soma
    \ENSURE True se existe uma combinação de elementos cuja soma seja igual a soma
    \STATE Function TemSoma(A, usados, inicio, fim, soma)
    \STATE $total \leftarrow 0$
    \FOR{$i = 0$ to n}
      \IF {$usados[i]$}
        \STATE $total \leftarrow total + A[i]$
      \ENDIF
    \ENDFOR
    \IF {$total = soma$}
      \RETURN \TRUE
    \ENDIF
    \FOR {$i = inicio$ to $fim$}
      \STATE $usados[i] \leftarrow 1$
      \IF {TemSoma(A, usados, i+1, fim, soma)}
        \RETURN \TRUE
      \ENDIF
      \STATE $usados[i] \leftarrow 0$
    \ENDFOR
    \RETURN \FALSE
    \STATE EndFunction
  \end{algorithmic}
\end{algorithm}


\subsection{Exercício 6}\label{sec:exer6}
\begin{enumerate}[a)]
  \item Gerar todas as permutações possíveis do conjunto a ser ordenado e verificar
  se os elementos estão em ordem.
  \item $\Theta(n!)$
\end{enumerate}


\subsection{Exercício 7}\label{sec:exer7}
Podemos iniciar uma DFS (ou BFS) por um vertice qualquer do grafo e marcar os vertices
visitados com 1. Quando a pilha (ou fila) ficar vazia, significa que todos os vertices
do componente conectado, e somente eles, foram visitados. Portanto, estão marcados com 1.
Se há vertices não visitados, reiniciar a travessia e marcada todos os vertices agora
visitados com 2. Repetir enquanto houver vertices não-visitados.

\subsection{Exercício 8}\label{sec:exer8}
\begin{enumerate}[a)]
  \item
  
\begin{tikzpicture}[->,>=stealth',shorten >=1pt,auto,node distance=0.5cm,
	thick,main node/.style={circle,fill=white!20,draw,minimum size=0.15cm,inner sep=0pt]}]
	
	\node[main node] (1) [fill=black] {};
	\node[main node] (2) [right of=1] {};
	\node[main node] (3) [above of=2] {};
	
	\node[main node] (4) [left of=3] {};
	\node[main node] (6) [above of=4] {};
	\node[main node] (8) [above of=6] {};
	\node[main node] (10) [above of=8] {};
	\node[main node] (12) [above of=10] {};
	
	\node[main node] (5) [right of=3] {};
	\node[main node] (7) [above of=5] {};
	\node[main node] (9) [above of=7] {};
	\node[main node] (11) [above of=9] {};
	\node[main node] (13) [above of=11] {};
	
	\node[main node] (14) [right of=8] {};
	\node[main node] (15) [right of=12] {};
	
	\node[main node] (16) [right of=13, fill=black]{};
	
	\path[-]
	(1) edge node {} (2)
	(2) edge node {} (3)
	(3) edge node {} (4)
	(3) edge node {} (5)
	
	(4) edge node {} (6)
	(6) edge node {} (8)
	(8) edge node {} (10)
	(10) edge node {} (12)
	
	(5) edge node {} (7)
	(7) edge node {} (9)
	(9) edge node {} (11)
	(11) edge node {} (13)
	
	(8) edge node {} (14)
	(9) edge node {} (14)
	
	(14) edge node {} (15)
	(12) edge node {} (15)
	(13) edge node {} (15)
	
	(13) edge node {} (16)
	;
	
	
	\end{tikzpicture}
  \item DFS. Porque nesse tipo de travessia caminha-se até encontrar um beco sem saída
  ou a saída do labirinto. Quando encontra-se um beco sem saída volta-se até o ponto
  imediatamente anterior que tenha opção de caminho não percorrido. No BFS, visita-se
  primeiro todos os vertices de um nível, para só então passar para o próximo nível.
  Portanto, a travessia de labirinto em DFS é mais eficiente.
\end{enumerate}

