%!TEX root = report.tex

\section{Exhaustive Search}\label{sec:exhaustive_force}

\subsection{Exercício 5}\label{sec:exer5}
Consideremos o seguinte conjunto A=[1, 5, 2, 3, 4, 5]. Para resolver o problema da
partição, podemos seguir os passos abaixo:
\begin{enumerate}
  \item Calcular a soma total dos elementos. No caso, resulta em 20.
  \item Calcular a metade (no caso será 10). Esse será o valor da soma de cada
  subconjunto.
  \item Achar uma combinação de elementos cuja soma seja 10. Isso pode ser feito
  com o algoritmo abaixo.
\end{enumerate}

\begin{algorithm}
  \caption{Determina se existe uma combinação de elementos cuja soma seja a desejada}
  \begin{algorithmic}
    \REQUIRE $A=[a_{0}, a_{1}, \cdots, a_{n}]$,$usados=[a_{0}, a_{1}, \cdots, a_{n}]$,
    inicio, fim e soma
    \ENSURE True se existe uma combinação de elementos cuja soma seja igual a soma
    \STATE Function TemSoma(A, usados, inicio, fim, soma)
    \STATE $total \leftarrow 0$
    \FOR{$i = 0$ to n}
      \IF {$usados[i]$}
        \STATE $total \leftarrow total + A[i]$
      \ENDIF
    \ENDFOR
    \IF {$total = soma$}
      \RETURN \TRUE
    \ENDIF
    \FOR {$i = inicio$ to $fim$}
      \STATE $usados[i] \leftarrow 1$
      \IF {TemSoma(A, usados, i+1, fim, soma)}
        \RETURN \TRUE
      \ENDIF
      \STATE $usados[i] \leftarrow 0$
    \ENDFOR
    \RETURN \FALSE
    \STATE EndFunction
  \end{algorithmic}
\end{algorithm}


\subsection{Exercício 6}\label{sec:exer6}

\subsection{Exercício 7}\label{sec:exer7}

\subsection{Exercício 8}\label{sec:exer8}
