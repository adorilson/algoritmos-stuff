%!TEX root = report.tex
\section{Dynamming Programming}\label{sec:dynamic_programming}

\subsection{Exercício 17}\label{sec:exer17}
Ambas técnicas dividem o problema em subproblemas menores. Na divisão-e-conquista,
os resultados dos subproblemas são combinados paa produzir o resultado o problema
original, ao passo que na programação dinâmica o(s) resultado(s) de subproblemas
são usados como "parâmetro" para a determinação dos seguintes até que se chegue
no resultado do problemas original.

\subsection{Exercício 18}\label{sec:exer18}

\lstinputlisting[language=Python]{../q18.py}

\subsection{Exercício 19}\label{sec:exer19}

\lstinputlisting[language=Python]{../q19.py}

\subsection{Exercício 20}\label{sec:exer20}

\lstinputlisting[language=Python]{../q20.py}

\subsection{Exercício 21 e 22}\label{sec:exer21}
O problema da mochila possui a seguite relação de recorrência:

$
f(i, j) =
\begin{cases} 
max \{F(i-1, j), v_i+F(i-1, j-w_i)\} \quad se\ j-w_i \ge 0,\\
F(i-1, j) \quad se\ j-w_i < 0
\end{cases}
$

As colunas da tabela da programação dinâmica representam a capacidade de mochila,
enquanto que as linhas representam os objetos. Assim, tanto o avanço para esquerda
em uma linha, quanto para baixo em uma linha significa a substituição de um objeto
por outro mais vantajoso ou a manutenção dos objetos até o momento. Portanto, em
ambos os casos, os valores são sempre não-decrescentes. Como no exemplo abaixo,
considerando uma mochila com capacidade igual a 5:

\begin{table}[ht]
    \begin{center}
      \begin{tabular}{c|rrrr}
& 1 & 2 & 3 & 4 \\
        \hline
w & 4   & 2   & 1   & 3 \\
v & 500 & 400 & 300 & 450 \\
      \end{tabular}
    \end{center}
    \caption{Pesos(w) e valores(v) dos objetos}
  \end{table}

\begin{table}[ht]
    \begin{center}
      \begin{tabular}{l|rrrrrr}
        \multicolumn{7}{c}{Capacidade j}\\
i & 0 & 1 & 2 & 3 & 4 & 5 \\
        \hline
0 & 0 & 0   & 0   & 0     & 0     & 0 \\
1 & 0 & 0   & 0   & 0     & 500   & 500 \\
2 & 0 & 0   & 400 & 400   & 500   & 500 \\
3 & 0 & 300 & 400 & 400   & 700   & 800 \\
4 & 0 & 300 & 400 & 450   & 750   & 850 \\
      \end{tabular}
    \end{center}
    \caption{Tabela preenchida}
  \end{table}
