%!TEX root = report.tex
\section{Dynamming Programming}\label{sec:dynamic_programming}

\subsection{Exercício 17}\label{sec:exer17}
Ambas técnicas dividem o problema em subproblemas menores. Na divisão-e-conquista,
os resultados dos subproblemas são combinados paa produzir o resultado o problema
original, ao passo que na programação dinâmica o(s) resultado(s) de subproblemas
são usados como "parâmetro" para a determinação dos seguintes até que se chegue
no resultado do problemas original.

\subsection{Exercício 18}\label{sec:exer18}

\subsection{Exercício 19}\label{sec:exer19}

\subsection{Exercício 20}\label{sec:exer20}

\subsection{Exercício 21}\label{sec:exer21}

\subsection{Exercício 22}\label{sec:exer22}
